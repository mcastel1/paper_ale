\documentclass[%
 reprint,
%superscriptaddress,
%groupedaddress,
%unsortedaddress,
%runinaddress,
%frontmatterverbose, 
%preprint,
%preprintnumbers,
%nofootinbib,
%nobibnotes,
%bibnotes,
 amsmath,amssymb,
 aps,
%pra,
%prb,
%rmp,
%prstab,
%prstper,
%floatfix,
]{revtex4-2}


% \input{/Users/michelecastellana/Documents/latex_modules/finite_elements/packages}
% \DeclareAcronym{bod}{
        short = {B},
        long = body,
        long-plural-form = bodies
}


\DeclareAcronym{bflu}{
        short = {F},
        long = bulk fluid,
        long-plural-form = bulk fluids
}

\DeclareAcronym{dom}{
        short = {D},
        long = domain}

% \input{/Users/michelecastellana/Documents/latex_modules/finite_elements/glossary}

% definitions for equations and  symbols
\providecommand{\mtitle}{Interaction between fluids and Helfrich membranes}
\providecommand{\fluid}{\textrm{F}}
\providecommand{\vf}{v_\fluid}
\providecommand{\rhof}{\rho_\fluid}
\providecommand{\etaf}{\eta_\fluid}
\providecommand{\sigmaf}{\sigma_\fluid}


% \externaldocument{supplementary}


% Ensure glossaries work
% \makenoidxglossaries  


% \author[1,2]{Michele Castellana\thanks{Corresponding author:  \href{michele.castellana@curie.fr}{michele.castellana@curie.fr}}}
% \affil[1]{\small{Institut Curie, PSL Research University, Paris, France}}
% \affil[2]{\small{CNRS UMR168, 11 rue Pierre et Marie Curie, 75005, Paris,France}}


\date{}


\begin{document}

\preprint{APS/123-QED}

% \setlength\intextsep{0pt}
% \setlength{\parskip}{5pt plus 0pt minus 0pt}

\title{\mtitle}% Force line breaks with \\
\thanks{A footnote to the article title}%

\author{Ann Author}
\altaffiliation[Also at ]{Physics Department, XYZ University.}%Lines break automatically or can be forced with \\
\author{Second Author}%
\email{Second.Author@institution.edu}
\affiliation{%
    Authors' institution and/or address\\
    This line break forced with \textbackslash\textbackslash
}%

\collaboration{MUSO Collaboration}%\noaffiliation

\author{Charlie Author}
\homepage{http://www.Second.institution.edu/~Charlie.Author}
\affiliation{
    Second institution and/or address\\
    This line break forced% with \\
}%
\affiliation{
    Third institution, the second for Charlie Author
}%
\author{Delta Author}
\affiliation{%
    Authors' institution and/or address\\
    This line break forced with \textbackslash\textbackslash
}%

\collaboration{CLEO Collaboration}%\noaffiliation

\date{\today}% It is always \today, today,
%  but any date may be explicitly specified

\begin{abstract}
    An article usually includes an abstract, a concise summary of the work
    covered at length in the main body of the article.
    \begin{description}
        \item[Usage]
              Secondary publications and information retrieval purposes.
        \item[Structure]
              You may use the \texttt{description} environment to structure your abstract;
              use the optional argument of the \verb+\item+ command to give the category of each item.
    \end{description}
\end{abstract}

%\keywords{Suggested keywords}%Use showkeys class option if keyword
%display desired
\maketitle



% \begin{appendices}
%   \input{text_appendix}
% \end{appendices}


% \glsaddall
% \printnoidxglossary

% \bibliographystyle{unsrt}
\bibliographystyle{unsrt-abbr}
\bibliography{/User/michelecastellana/Documents/my_bibliography/bibliography}


\end{document}

