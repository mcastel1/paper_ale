\section{Fluid and rigid body}\label{sec_fl_rigid}

In this Section, we will discuss the interaction between a bulk \ac{fl} and the simplest structure---a rigid \ac{bod} which is only allowed to turn about one point.






All quantities relative to the \acl{fl}, \acl{bod} and \acl{dom} will be denoted by the suffix \ac{fl}, \ac{bod} and \acl{dom}, respectively.

In all the systems that we will consider, we will introduce the \ac{ref} and \ac{cur} configuration \cite{kamenskyLectureNotesMAE,landauTheoryElasticity1986,slaughterLinearizedTheoryElasticity2002}, see \cref{fig_ref_curr}.
In the \ac{ref} configuration, the region occupied by the bulk fluid is described by coordinates $y^1,y^2$. The  axis of the ellipse, e..g, the rigid body, lies parallel to the $y^1$ axis.
In the \ac{cur} configuration, the region occupied by the bulk fluid is described by coordinates $x^1, x^2$. The  axis of the ellipse, e..g, the rigid body, forms an angle  $\theta$ (red arc) with respect to the $x^1$ axis. All quantities relative to the \ac{ref} and \ac{cur} configuration will be denoted by the superscript \ac{ref} and \ac{cur}, respectively.

%sign

\begin{figure*}
    \centering
    \includegraphics[width=0.8\textwidth]{figures/figure_6/figure_6.pdf}
    \caption{
        \acresetall
        \label{fig_ref_curr}
        Reference and current configuration for the interaction between a bulk fluid and a rigid body.
        \plab{A} Reference configuration: Boundaries in the reference configuration are denoted by $\pomineqr$,$\pomouteqr$, $\pomtopeqr$, $\pombottomeqr$ and $\pomellipseeqr$ (colored dashed lines). The mesh in the reference configuration (gray lines) is also shown.
        \plab{B} Current configuration. Boundaries in the current configuration are denoted by $\pomineqc$,$\pomouteqc$, $\pomtopeqc$, $\pombottomeqc$ and $\pomellipseeqc$ (colored dashed lines), and the rotational angle $\theta$ of the rigid body is shown as an arc (black dotted line). The mesh in the current configuration (gray lines) is also shown.
    }
\end{figure*}



The equations of motion, expressed for the current configuration, are


\begin{align}
    \label{eq_fl_rigid_fl_1}
    \rho_\fluid \left( \partial_t \vf^\alpha + \vf^\beta \partial^x_\beta \vf^\alpha \right) = & \partial^x_\alpha  \sigma_\fluid + \eta_\fluid \partial^x_\beta \partial^x_\beta \vf^\alpha, \\
    \label{eq_fl_rigid_fl_2}
    \partial^x_\alpha \vf^\alpha =                                                             & 0                               ,                                                            \\
    \label{eq_fl_rigid_rigid}
    I \frac{\dint{}^2 \theta}{\dint{} t^2} =                                                   & - \int_\pomellipseeqc \, \epsilon_{3 \alpha \beta} \sigma_{\beta \gamma} n_\gamma \dint{l}.
\end{align}

Here,  \cref{eq_fl_rigid_fl_1,eq_fl_rigid_fl_2} are the equations of motion for  \ac{fl}, and \cref{eq_fl_rigid_rigid} is the equation of motion for the \ac{bod}.

\Cref{eq_fl_rigid_fl_1,eq_fl_rigid_fl_2} are the \ac{ns} and continuity equation, respectively, in a flat, two-dimensional geometry \cite{landauFluidMechanics1987}. There, $\rhof$ and $\etaf$, $\vf^\alpha$ and $\sigmaf$ are the two-dimensional density and viscosity,  velocity and surface tension  of  \ac{fl}, respectively, and
\be\label{eq_def_pder}
\partial^x_\alpha \equiv \pder{}{x^\alpha}.
\ee

Here, greek indices $\alpha, \beta, \cdots$ will be used to denote vectors and tensors in  two-dimensional Euclidean space---their position as upper or lower indices is thus immaterial \cite{marchiafavaAppuntiDiGeometria2005}.

\Cref{eq_fl_rigid_rigid} is the rigid-body equation of motion \cite{landauMechanics1960}, which relates the angular acceleration in the \ac{lhs}, to the momentum of external forces exerted by the fluid on the body in the \ac{rhs} \cite{landauMechanics1960}. In there, $I$ is the moment of inertia of the body. The \ac{rhs} contains the integral along the surface of the rigid body, of the forces exerted by \ac{fl} on  \ac{bod} \cite{landauFluidMechanics1987}. These forces are expressed in terms of the fluid stress tensor
\be
\label{eq_strees_tensor_fluid}
\sigma_{\alpha \beta } \equiv \sigma_\fluid \delta_{\alpha \beta}+ \eta_\fluid\left( \partial^x_\beta v_\fluid^\alpha + \partial^x_\alpha v_\fluid^\beta \right),
\ee
where $\epsilon_{\alpha \beta \gamma}$ is the three-dimensional Levi-Civita symbol. Finally, $\dint{l}$ is the line element of the boundary of  \ac{bod}, and $n_\alpha$ the boundary normal.

We will now introduce the reference configuration, and relate it to the current one \cite{kamenskyLectureNotesMAE}, see \cref{fig_ref_curr}. At a time $t$, the point $\bx$ in the current configuration corresponds to a point $\by$ in the reference configuration through the deformation field
\be\label{eq_def_u}
u^{\alpha \, t}(\by) \equiv x^\alpha - y^\alpha.
\ee
and the deformation-gradient tensor
\be
\label{eq_def_F}
F_{\alpha \beta} \equiv \delta_{\alpha \beta} +\partial^y_\beta u^{t \, \alpha}.
\ee
%sign

The \acl{bc} for \cref{eq_fl_rigid_fl_1,eq_fl_rigid_fl_2,eq_fl_rigid_rigid} are
\begin{align}
    \vf^1                                     & =  v_\ast \; \text{ on } \pomineq\label{bc_fl_rigid_1},            \\
    \vf^2                                     & =  0 \; \text{ on } \pomineq \label{bc_fl_rigid_2},                \\
    {\bm \vf}                                 & = \bf{0} \; \text{ on } \pomweq,                                   \\
    \vf^\alpha                                & = \der{R(\theta)_{\alpha \beta} }{t} [\psi^t_\beta(\bx) - c_\beta] \\
    \hat{n}_\beta \partial^x_\beta \vf^\alpha & = 0  \; \text{ on } \pomouteq,                                     \\
    \sigmaf                                   & = 0 \; \text{ on } \pomouteq
\end{align}

\begin{figure*}
    \centering
    \includegraphics[width=\textwidth]{figures/figure_1/figure_1.pdf}
    \caption{
        \label{fig_fluid_rigid}
        Interaction between a bulk fluid and a rigid body. \plab{A} Temporal snapshot of the mesh (black lines) and velocity field (arrows). The direction of the velocity field is indicated by the arrows, and its norm by their color. The rigid body (ellipse) is allowed to pivot about its left focal point (red dot), and the related angle with respect to the $x$ direction is denoted by the red arc.   \plab{B} Color map of the surface tension. Both panels refer to the same instant of time, shown on top.
    }
\end{figure*}