\section{Fluid and rigid body}\label{sec_fl_rigid}

In this Section, we will discuss the interaction between a  \ac{bflu} and the simplest structure---a rigid \ac{bod} which is only allowed to turn about one point.






% All quantities relative to the \ac{bflu}, \ac{bod} and \ac{dom} will be denoted by the suffix \ac{bflu}, \ac{bod} and \ac{dom}, respectively.

In all the systems that we will consider, we will introduce the \ac{ref} and \ac{cur} configuration \cite{kamenskyLectureNotesMAE,landauTheoryElasticity1986,slaughterLinearizedTheoryElasticity2002}, see \cref{fig_ref_curr}. All quantities relative to the \ac{ref} and \ac{cur} configuration will be denoted by the superscript \ac{ref} and \ac{cur}, respectively.
In the \ac{ref} configuration, the region occupied by the bulk fluid is described by Cartesian coordinates $\bxr$. The  axis of the ellipse, e..g, the rigid body, lies parallel to the $\xc^1$ axis.
In the \ac{cur} configuration, the region occupied by the bulk fluid is described by Cartesian coordinates $\bxc$. The  axis of the ellipse, e..g, the rigid body, forms an angle  $\theta$ (red arc) with respect to the $\xc^1$ axis.


The \ac{ref} configuration, is related to the \ac{cur} one as follows \cite{kamenskyLectureNotesMAE}, see \cref{fig_ref_curr}. At  time $t$, the point $\bxc$  corresponds to a point $\bxr$ in the reference configuration through the deformation field \cite{landauFluidMechanics1987}
\be\label{eq_def_u}
u^{\alpha \, t}(\bxr) \equiv \xc^\alpha - \xr^\alpha.
\ee
and the deformation-gradient tensor
\be
\label{eq_def_F}
F_{\alpha \beta} \equiv \delta_{\alpha \beta} + \pderr{u^{t \, \alpha}}{\beta},
\ee
where we set
\begin{align}\label{eq_def_pder_ref}
        \partial^\reference_\alpha & \equiv \pder{}{\xr^\alpha}, \\
        \label{eq_def_pder_cur}
        \partial^\current_\alpha   & \equiv \pder{}{\xc^\alpha}.
\end{align}

Here, we will denote the mapping between $\bxc$ and $\bxr$ in \cref{eq_def_u} with the fields
\begin{align}\label{eq_def_phi}
        \bxc & = \bm{\phi}^t(\bxr), \\
        \label{eq_def_psi}
        \bxr & = \bm{\psi}^t(\bxc).
\end{align}

We denote the \ac{bflu} velocity and surface tension by $\vfc(\bxc)$ and $\sigmafc(\bxc)$, respectively, where the superscript \ac{cur} specifies that these fields depend on the coordinate $\bxc$ in the \ac{cur} configuration.


\begin{figure*}
        \centering
        \includegraphics[width=\textwidth]{figures/figure_6/figure_6.pdf}
        \caption{
                \acresetall
                \label{fig_ref_curr}
                \Ac{ref} and \ac{cur} configuration for the interaction between a bulk fluid and a rigid body.
                %
                \plab{A} \Ac{ref} configuration. The region $\omr$ is delimited by the mesh (gray lines).  Boundaries in the \ac{ref} configuration are denoted by $\pomineqr$,$\pomouteqr$, $\pomtopeqr$, $\pombottomeqr$ and $\pomellipseeqr$ (colored dashed lines). The Cartesian coordinates $\bxr$  are also shown.
                %
                \plab{B} \Ac{cur} configuration. The region $\omc$ is delimited by the mesh (gray lines). Boundaries in the \ac{cur} configuration are denoted by $\pomineqc$,$\pomouteqc$, $\pomtopeqc$, $\pombottomeqc$ and $\pomellipseeqc$ (colored dashed lines), and the rotational angle $\theta$ of the rigid body is shown as an arc (black dotted line). The Cartesian coordinates are also shown.
        }
\end{figure*}

\subsection{Rigid-body motion}\label{sec_rigid_body_body}

The dynamics of the \ac{bod} is given by the equations of motion for a rigid body \cite{goldsteinClassicalMechanics2004}

\begin{align}
        \label{eq_fl_rigid_rigid_current_1}
        I         \der{\omega}{t}
               & =                                                                                             - \int_\pomellipseeqc \dint{l_\current} \, \epsilon_{3 \alpha \beta} \, \sigmastressc_{\beta \gamma} \hat{n}_\gamma \dint{l_\current}, \\
        \label{eq_fl_rigid_rigid_current_2}
        \omega & \equiv \der{\theta}{t},
\end{align}
which relate the angular acceleration of \ac{bod} to the momentum of external forces.
Greek indices $\alpha, \beta, \cdots$ will be used to denote vectors and tensors in  two-dimensional Euclidean space---their position as upper or lower indices is thus immaterial \cite{marchiafavaAppuntiDiGeometria2005,landauTheoryElasticity1986}.
The moment of inertia of the \ac{bod} is denoted by $I$. The \ac{rhs} of \cref{eq_fl_rigid_rigid_current_1} contains the line integral along the boundary of \ac{bod}, of the torque of the forces exerted by the \ac{bflu} on  \ac{bod} with respect to the left focal point $\bm c$ of \ac{bod} \cite{landauFluidMechanics1987}. These forces are expressed in terms of the fluid stress tensor \cite{landauFluidMechanics1987}
\be
\label{eq_strees_tensor_fluid}
\sigmastressc_{\alpha \beta } \equiv \sigmafc \delta_{\alpha \beta}+ \eta_\fluid\left( \pderc{ \vfc^\alpha}{\beta} +\pderc{ \vfc^\beta}{\alpha}\right),
\ee
where $\epsilon_{\alpha \beta \gamma}$ is the three-dimensional Levi-Civita symbol, and $\etaf$ the two-dimensional viscosity of \ac{bflu}. Finally,   $\dint{l_\current}$ is the line element of \pomellipsec in the \ac{cur} configuration, and $\hat{n}$ the unit boundary normal to \pomellipsec pointing outside $\omc$.



\subsection{Fluid motion}\label{sec_rigid_body_fl}



\subsubsection{Equations of motion in the current-configuration coordinates}\label{sec_rigid_body_fl_cur}



The equations of motion for \ac{bflu}, expressed in terms of the \ac{cur} fields $\vfc$ and $\sigmafc$, are


\begin{align}
        \label{eq_fl_rigid_fl_1_current}
        \rho_\fluid \left( \partial_t \vfc^\alpha + \vfc^\beta  \pderc{\vfc^\alpha}{\beta} \right) & =  \pderc{  \sigmafc}{\alpha} + \eta_\fluid \pder{}{\xc^\beta} \pder{\vfc^\alpha}{\xc^\beta} ,   \\
        \label{eq_fl_rigid_fl_2_current}
        \pderc{ \vfc^\alpha }{\alpha}                                                              & =                                                              0                               ,
\end{align}
and they hold for $\xc \in \omc$. \Cref{eq_fl_rigid_fl_1_current,eq_fl_rigid_fl_2_current} are the \ac{ns} and continuity equation, respectively, in a flat, two-dimensional geometry \cite{landauFluidMechanics1987}. There, $\rhof$ is  the two-dimensional \ac{bflu} density.
%



The \acp{bc} for \cref{eq_fl_rigid_fl_1_current,eq_fl_rigid_fl_2_current,eq_fl_rigid_rigid_current_1,eq_fl_rigid_rigid_current_2} are
\begin{align}
        \label{bc_fl_rigid_1_current}
        \vfc^1(\bxc)                  & =  v_\ast \; \text{ on } \pomineqc,                                                                        \\
        \label{bc_fl_rigid_2_current}
        \vfc^2  (\bxc)                & =  0 \; \text{ on } \pomineqc ,                                                                            \\
        \label{bc_fl_rigid_3_current}
        {\bm \vfc}    (\bxc)          & = \bzero \; \text{ on } \pomweqc ,                                                                         \\
        \label{bc_fl_rigid_4_current}
        \vfc^\alpha (\bxc)            & = \der{[R(\theta)]_{\alpha \beta} }{t} [\psi^{t \, \beta}(\bxc) - c^\beta] \; \text{ on } \pomellipseeqc , \\
        \label{bc_fl_rigid_5_current}
        \etaf \pderc{ \vfc^\alpha}{1} & = 0  \; \text{ on } \pomouteqc ,                                                                           \\
        \label{bc_fl_rigid_6_current}
        \sigmafc (\bxc)               & = 0 \; \text{ on } \pomouteqc.
\end{align}
\Cref{bc_fl_rigid_1_current,bc_fl_rigid_2_current} ensure that the fluid is injected on the boundary \pominc{} in a direction parallel to the $\xc^1$ axis, and \cref{bc_fl_rigid_3_current} is a no-slip \ac{bc} \cite{landauFluidMechanics1987} along the channel walls \pomwc, see \cref{fig_ref_curr}.
\Cref{bc_fl_rigid_4_current} ensures that, at the boundary \pomellipsec,  the \ac{bflu} velocity matches the rotational velocity of \ac{bod}. Given that \ac{bod} can only rotate about its left focal point $\bm c$, the \ac{bod} rotational velocity of a point $\bxc \in \pomellipseeqc$ is given by
\be
\der{[R(\theta)]_{\alpha \beta}}{t}[\psi^{t \, \beta}(\bxc) - c_\beta],
\ee
where $\bm R$ is the  rotation matrix corresponding to a counterclockwise rotation:
\be
{\bm R}(\theta) \equiv
\left(
\begin{matrix}
                \cos \theta & - \sin \theta \\
                \sin \theta & \cos \theta
        \end{matrix}
\right).
\ee
Finally, \cref{bc_fl_rigid_5_current,bc_fl_rigid_6_current} ensure, respectively, that on \pomoutc there is zero traction and tension; this corresponds to the hypothesis that there is free flow at \pomoutc \cite{loggAutomatedSolutionDifferential2012,landauFluidMechanics1987}.

\subsubsection{Equations of motion in the reference-configuration coordinates}\label{sec_rigid_body_fl_ref}

The solution of the \ac{bvp} \crefs{eq_fl_rigid_fl_1_current,eq_fl_rigid_fl_2_current,eq_fl_rigid_rigid_current_1,eq_fl_rigid_rigid_current_2,bc_fl_rigid_1_current,bc_fl_rigid_2_current,bc_fl_rigid_3_current,bc_fl_rigid_4_current,bc_fl_rigid_5_current,bc_fl_rigid_6_current} is  involved because of the presence of the moving boundary \pomellipsec, cf. \cref{fig_ref_curr}. Following the \ac{ale} procedure, we  bridge between the Lagrangian and the Eulerian description used to describe, respectively,  \ac{bod} and \ac{bflu}  \cite{kamenskyLectureNotesMAE}.


For each field ${\bm \vfc}({\bxc}), \sigmaf(\bxc)$ which depends on the coordinate $\bxc$ in the \ac{cur} configuration, we introduce its counterpart in the \ac{ref} configuration:

\begin{align}
        \label{eq_vf_ref}
        {\bm \vfc}(\bxr, t) & \equiv {\bm \vfr}( \bm{\phi}^t(\bxr), t),     \\
        \label{eq_sigmaf_ref}
        \sigmafc(\bxr, t)   & \equiv {\bm \sigmafr}( \bm{\phi}^t(\bxr), t).
\end{align}

We will now rewrite the equations of motion and the \acp{bc}   in terms of ${\bm \vfc}$, $\sigmafc$, and the \ac{ref} coordinate $\bxr$. The equations of motion \crefs{eq_fl_rigid_fl_1_current,eq_fl_rigid_fl_2_current,eq_fl_rigid_rigid_current_1} read
\bw
\begin{align}
        \label{eq_fl_rigid_fl_1_reference}
        \rhof \left\{ \pder{ \vfr^\alpha(\bxr, t)}{t} + \left[ \vfr^\gamma(\bxr, t)  - \der{\phi^{\gamma\, t}(\bxr)}{t}\right] G^t_{\beta \gamma}(\bxr) \frac{\partial  \vfr^\alpha(\bxr, t)}{\partial \xr^\beta} \right\}                                            & =              \newlinenn
        G^t_{\beta \alpha}(\bxr)  \,          \pderr{ \sigmafr(\bxr, t)   }{\beta }                + \etaf \,              G^t_{\delta \beta}(\bxr)            \,         \pderr{}{\delta} \left[  G^t_{\gamma \beta}(\bxr) \pderr{ \vfr(\bxr, t)}{\gamma}   \right], &                                                            \\
        \label{eq_fl_rigid_fl_2_reference}
        G^t_{\beta \alpha}(\bxr) \pderr{ \vfr^\alpha(\bxr, t)}{\beta}                                                                                                                                                                                                 & = 0,                                                       \\
        \label{eq_fl_rigid_rigid_reference}
        I \der{\omega}{t}                                                                                                                                                                                                                                             & =                                               \newlinenn
        \int_\pomellipseeqr \dint{s} \, \epsilon_{\alpha \beta} \left[ \phi^{\alpha \, t}(\bxi{s}) - c^\alpha \right] \sigmastressr_{\beta \gamma}(\bxi{s}) \epsilon_{\gamma \lambda} F^t_{\lambda \mu}(\bxi{s}) \der{\xi^\mu(s)}{s}                                        ,
\end{align}
\ew
where \cref{eq_fl_rigid_fl_1_reference,eq_fl_rigid_fl_2_reference,eq_fl_rigid_rigid_reference} hold in $\omr$.

Where the inverse of the deformation-gradient tensor is
\be
\label{eq_def_G}
G^t_{\alpha \beta}(\bxr) \equiv [F^t(\bxr)]^{-1}_{\alpha \beta},
\ee
and the stress tensor in the \ac{ref} configuration is
\be
\label{eq_def_sigma_current}
\sigmastressr_{\alpha \beta}^t \equiv  \sigmafr \delta_{\alpha \beta} + \etaf \left[ G^t_{\gamma \beta} \pderr{\vfc^\alpha}{\gamma} + G^t_{\gamma \alpha} \pderr{\vfc^\beta}{\gamma}\right].
\ee
In \cref{eq_fl_rigid_rigid_reference}, the integral in the \ac{rhs} is over the curvilinear boundary \pomellipser, which is parametrized as $\bxr = \bxi{s}$, where $s$ is the curvilinear coordinate, and $\epsilon_{\alpha \beta}$ is the two-dimensional Levi-Civita symbol \cite{marchiafavaAppuntiDiGeometria2005}.

The \acp{bc}  \crefs{bc_fl_rigid_1_current,bc_fl_rigid_2_current,bc_fl_rigid_3_current,bc_fl_rigid_4_current,bc_fl_rigid_5_current,bc_fl_rigid_6_current} can be rewritten as

\begin{align}
        \label{bc_fl_rigid_1_reference}
        \vfr^1(\bxr)                                          & =  v_\ast \; \text{ on } \pomineqr,                                                          \\
        \label{bc_fl_rigid_2_reference}
        \vfr^2      (\bxr)                                    & =  0 \; \text{ on } \pomineqr ,                                                              \\
        \label{bc_fl_rigid_3_reference}
        {\bm \vfr} (\bxr)                                     & = {\bm 0} \; \text{ on } \pomweqr ,                                                          \\
        \label{bc_fl_rigid_4_reference}
        \vfr^\alpha(\bxr)                                     & = \der{[R(\theta)]_{\alpha \beta} }{t} [\xr^\beta - c^\beta] \; \text{ on } \pomellipseeqr , \\
        \label{bc_fl_rigid_5_reference}
        \etaf G^t_{\beta 1}(\bxr) \pderr{ \vfr^\alpha}{\beta} & = 0  \; \text{ on } \pomouteqr ,                                                             \\
        \label{bc_fl_rigid_6_reference}
        \sigmafr (\bxr)                                       & = 0 \; \text{ on } \pomouteqr.
\end{align}

Overall, \cref{\eqsfluidbodyref} depend on the \ac{ref} fields and coordinate only, and they constitute, at any time $t$, a  \ac{bvp} which, combined with the temporal \acp{bc},
\begin{align}
        {\bm \vfr}(\bxr, t=0)     & = {\bm \vfr}_0(\bxr),     \\
        {\bm \sigmafr}(\bxr, t=0) & = {\bm \sigmafr}_0(\bxr), \\
\end{align}
determines $\bm \vfr$ and $\sigmafr$ as functions of space and time.

\subsection{Fluid-domain motion}
\label{sec_eq_fluid_domain}

The equations of motion \crefs{\eqsfluidbodyref} involve the displacement field $\bm u$, which will be determined with an analogy with the theory of elasticity. We imagine, as \ac{bod} rotates about its focal point, that each point $\bxr \in \omr$ moves into a point $\bxc \in \omc$, and that the domain $\omr$ is deformed, e.g., locally stretched and compressed, into $\omc$, as an elastic body.

The \ac{bvp} which determines $\bm u$ is given by the following elastic model \cite{kamenskyLectureNotesMAE,marsdenMathematicalFoundationsElasticity1994}

\be\label{eq_mesh_fluid_rigid}
\pder{S_{\alpha \beta}}{\xr^\beta} = 0 \text{ in } \omr,
\ee
where the elastic stress tensor $S$ is given by
\be
S_{\alpha \beta} \equiv F_{\alpha \gamma} T_{\gamma \beta},
\ee
and
\begin{align}
        \label{eq_ela_1}
        T_{\alpha \beta} & \equiv \kel(\bu) E_{\gamma \gamma} \delta_{\alpha \beta} +2 \muel(\bu) \left[E_{\alpha \beta} - \frac{1}{2} \delta_{\alpha \beta} E_{\gamma \gamma}\right], \\
        \label{eq_ela_2}
        E_{\alpha \beta} & \equiv \frac{1}{2}\left( F_{\gamma \alpha}F_{\gamma \beta} - \delta_{\alpha \beta}\right),                                                                  \\
        \label{eq_ela_3}
        \kel(\bu)        & \equiv \frac{1}{[\det(F(\bu))]^\zeta},                                                                                                                      \\
        \label{eq_ela_4}
        \muel(\bu)       & \equiv \frac{1}{[\det(F(\bu))]^\zeta}.
\end{align}
In \cref{eq_ela_3,eq_ela_4}, we included an additional dependence of bulk modulus $\kel$ and the modulus of compression $\muel$ on the deformation-gradient tensor \crefs{eq_def_F}. In the absence of this dependence, the elastic model above would be unstable under local compression. In fact, the model's Lagrangian would not penalize configurations with  small $\det F$. The inclusion of the dependence \crefs{eq_ela_3,eq_ela_4} on $\det F$  allows for penalizing these configurations, and it makes the model stable under mesh compression, such as the one below the ellipse in \cref{fig_ref_curr}B.

The \acp{bc} for the  \ac{pde} \crefs{eq_mesh_fluid_rigid} are
\begin{align}
        \label{bc_ela_1}
        \bu & =\bzero \text{ on } \pomsqeqr,                                                       \\
        \label{bc_ela_2}
        \bu & = {\bm c} - \bxr + {\bm R}(\theta)\cdot (\bxr - {\bm c}) \text{ on } \pomellipseeqr.
\end{align}
\Cref{bc_ela_1} enforces the fact that the mesh is not deformed at the boundary \pomsqr. Finally,  \cref{bc_ela_2} ensures that the mesh deformation at \pomellipser  matches the deformation induced by the rotation of \ac{bod}.

Overall, \cref{eq_mesh_fluid_rigid,bc_ela_1,bc_ela_2} yield a \ac{bvp} which determines $\bu$.

\subsection{Temporal discretization}\label{sec_temp_discretization}


To numerically solve for the dynamics of the system in a time span $0 \leq t \leq T$, we discretize time by setting $\deltat \equiv T/N$, $t_n \equiv n\, \deltat$, with $n = 0, 1, \cdots, N$. We set
\begin{align}
        \sigmafr^n(\bxr)       & \equiv \sigmafc(\bxr, t_n)                               \\
        \sigmafr^{n-1/2}(\bxr) & \equiv \sigmafc\left(\bxr, \frac{t_n+t_{n-1}}{2}\right),
\end{align}
and similarly for other quantities. In general,  the superscript $n$ assigned to quantity which depends on multiple fields, e.g., $\sigmastressr_{\alpha \beta}^n$, means that all such fields are evaluated at $t = t_n$.

We combine the \ac{cn} discretization method for \ac{ns} equations \cite{loggAutomatedSolutionDifferential2012,crankPracticalMethodNumerical1947} with the \ac{ale} formulation as follows. We discretize \cref{eq_fl_rigid_rigid_reference,eq_fl_rigid_rigid_current_2} and, neglecting $\odeltat$, we obtain
\bw
\begin{align}
        \label{eq_def_omega_disc}
        \frac{\theta^n - \theta^{n-1}}{\deltat}     = & \omega^n       ,                                                                                                                                                 \\
        \label{eq_fl_rigid_rigid_reference_disc}
        \frac{\omega^n - \omega^{n-1}}{\deltat}  =    &
        I  \int_\pomellipseeqr \dint{s} \, \epsilon_{\alpha \beta} \left[ \xi^\alpha(s) + u^{n-1,\, \alpha}(\bxi{s})- c^\alpha \right] \times                                                                            \newlinenn
                                                      & \sigmastressr^{n-1}_{\beta \gamma}(\bxi{s}; {\bm \vfr^{n-1}, \sigmafr^{n-3/2}}) \epsilon_{\gamma \lambda} F^{n-1}_{\lambda \mu}(\bxi{s}) \der{\xi^\mu(s)} {s}  ,
\end{align}
\ew
where in \cref{eq_fl_rigid_rigid_reference_disc} we explicitly indicated the dependence of the stress tensor \crefs{eq_def_sigma_current} on $\bm \vfc$ and $\sigmafc$.
The discrete version of  the \ac{bvp} \cref{eq_mesh_fluid_rigid,bc_ela_1,bc_ela_2} is
\begin{align}
        \label{eq_mesh_fluid_rigid_disc}
        \pder{S_{\alpha \beta}(\bu^n)}{\xr^\beta} & = 0 \text{ in } \omr,                                                                  \\
        \label{bc_ela_1_disc}
        \bu^n                                     & =\bzero \text{ on } \pomsqeqr,                                                         \\
        \label{bc_ela_2_disc}
        \bu^n                                     & = {\bm c} - \bxr + {\bm R}(\theta^n)\cdot (\bxr - {\bm c}) \text{ on } \pomellipseeqr.
\end{align}

At each step $n$
\begin{enumerate}
        \item \label{fl_rigid_step_1}
              Given $\theta^{n-1}, \omega^{n-1}, {\bm \vfr}^{n-1}, \sigmafr^{n-3/2}, \bu^{n-1}$ from the preceding step,
              %sign
        \item \label{fl_rigid_step_2}
              We solve \cref{eq_def_omega_disc,eq_fl_rigid_rigid_reference_disc} for $\theta^n, \omega^n$,
        \item \label{fl_rigid_step_3}

\end{enumerate}


\begin{figure*}
        \centering
        \includegraphics[width=\textwidth]{figures/figure_1/figure_1.pdf}
        \caption{
                \label{fig_fluid_rigid}
                Interaction between a bulk fluid and a rigid body. \plab{A} Temporal snapshot of the mesh (black lines) and velocity field (arrows). The direction of the velocity field is indicated by the arrows, and its norm by their color. The rigid body (ellipse) is allowed to pivot about its left focal point (red dot), and the related angle with respect to the $x$ direction is denoted by the red arc.   \plab{B} Color map of the surface tension. Both panels refer to the same instant of time, shown on top.
        }
\end{figure*}