\section{Fluid and rotating rigid body}\label{sec_fl_rigid}

In this Section, we will discuss the interaction between a  \ac{bflu} and the simplest structure---a rigid \ac{bod} which is only allowed to turn about one point.


In all the systems that we will consider, we will introduce the \ac{ref} and \ac{cur} configuration \cite{kamenskyLectureNotesMAE,landauTheoryElasticity1986,slaughterLinearizedTheoryElasticity2002}, see \cref{fig_ref_curr}. All quantities relative to the \ac{ref} and \ac{cur} configuration will be denoted by the superscript \ac{ref} and \ac{cur}, respectively.
In the \ac{ref} configuration, the region occupied by the bulk fluid is described by Cartesian coordinates $\bxr$. The  axis of the ellipse, e..g, the rigid body, lies parallel to the $\xc^1$ axis.
In the \ac{cur} configuration, the region occupied by the bulk fluid is described by Cartesian coordinates $\bxc$. The  axis of the e`llipse, e..g, the rigid body, forms an angle  $\theta$ (red arc) with respect to the $\xc^1$ axis.


The \ac{ref} configuration, is related to the \ac{cur} one as follows \cite{kamenskyLectureNotesMAE}, see \cref{fig_ref_curr}. At  time $t$, the point $\bxc$  corresponds to a point $\bxr$ in the reference configuration through the deformation field \cite{landauFluidMechanics1987}
\be\label{eq_def_u}
u^{\alpha \, t}(\bxr) \equiv \xc^\alpha - \xr^\alpha.
\ee
and the deformation-gradient tensor
\be
\label{eq_def_F}
F_{\alpha \beta} \equiv \delta_{\alpha \beta} + \pderr{u^{t \, \alpha}}{\beta},
\ee
where we set
\begin{align}\label{eq_def_pder_ref}
        \partial^\reference_\alpha & \equiv \pder{}{\xr^\alpha}, \\
        \label{eq_def_pder_cur}
        \partial^\current_\alpha   & \equiv \pder{}{\xc^\alpha}.
\end{align}

Here, we will denote the mapping between $\bxc$ and $\bxr$ in \cref{eq_def_u} with the fields
\begin{align}\label{eq_def_phi}
        \bxc & = \bm{\varphi}^t(\bxr), \\
        \label{eq_def_psi}
        \bxr & = \bm{\psi}^t(\bxc).
\end{align}

We denote the \ac{bflu} velocity and surface tension by $\vfc(\bxc)$ and $\sigmafc(\bxc)$, respectively, where the superscript \ac{cur} specifies that these fields depend on the coordinate $\bxc$ in the \ac{cur} configuration.


The solution of the interaction dynamics between \ac{bflu} and \ac{bod} is  involved because of the presence of the moving boundary \pomellipsec, cf. \cref{fig_ref_curr}. Following the \ac{ale} procedure, we will  bridge between the Lagrangian and the Eulerian description used to describe, respectively,  \ac{bod} and \ac{bflu}  \cite{kamenskyLectureNotesMAE}. To achieve this, in what follows we will write the equations of motion for \ac{bod} and \ac{bflu}, which are naturally formulated the \ac{cur} configuration first, and then rewrite them in the \ac{ref} configuration.


\begin{figure*}
        \centering
        \includegraphics[width=\textwidth]{figures/figure_6/figure_6.pdf}
        \caption{
                \label{fig_ref_curr}
                \Ac{ref} and \ac{cur} configuration for the interaction between a bulk fluid and a rigid body.
                %
                \plab{A} \Ac{ref} configuration. The region $\omr$ is delimited by the mesh (gray lines).  Boundaries in the \ac{ref} configuration are denoted by $\pomineqr$,$\pomouteqr$, $\pomtopeqr$, $\pombottomeqr$ and $\pomellipseeqr$ (colored dashed lines). The Cartesian coordinates $\bxr$  are also shown.
                %
                \plab{B} \Ac{cur} configuration. The region $\omc$ is delimited by the mesh (gray lines). Boundaries in the \ac{cur} configuration are denoted by $\pomineqc$,$\pomouteqc$, $\pomtopeqc$, $\pombottomeqc$ and $\pomellipseeqc$ (colored dashed lines), and the rotational angle $\theta$ of the rigid body is shown as an arc (black dotted line). The Cartesian coordinates are also shown.
        }
\end{figure*}

\subsection{Rigid-body motion}\label{sec_rigid_body_body}

In this Section, we will work out the equations of motion for \ac{bod}.

\subsubsection{Equation of motion in the \acl{cur}-configuration coordinates}\label{sec_rigid_body_body_current}

The dynamics of  \ac{bod} is given by the equations of motion for a rigid body which, in \ac{cur} coordinates, read \cite{goldsteinClassicalMechanics2004}

\begin{align}
        \label{eq_fl_rigid_rigid_current_1}
        I         \der{\omega}{t}
               & =                                                                                             - \int_\pomellipseeqc \dint{l_\current} \, \epsilon_{3 \alpha \beta} \, \sigmastressc_{\beta \gamma} \hat{\bm n}^{\current\,\gamma} , \\
        \label{eq_fl_rigid_rigid_current_2}
        \omega & \equiv \der{\theta}{t}.
\end{align}
\Cref{eq_fl_rigid_rigid_current_1} relates the angular acceleration of \ac{bod} to the torque of external forces.
In what follows, Greek indices $\alpha, \beta, \cdots$ will be used to denote vectors and tensors in  two-dimensional Euclidean space---their position as upper or lower indices is thus immaterial \cite{marchiafavaAppuntiDiGeometria2005,landauTheoryElasticity1986}.
The moment of inertia of the \ac{bod} is denoted by $I$. The \ac{rhs} of \cref{eq_fl_rigid_rigid_current_1} contains the line integral along the boundary of \ac{bod}, of the torque of the local force exerted by  \ac{bflu} on  \ac{bod} with respect to the left focal point $\bfocalpoint$ of \ac{bod} \cite{landauFluidMechanics1987}. This force is expressed in terms of the \ac{bflu} stress tensor \cite{landauFluidMechanics1987}
\be
\label{eq_strees_tensor_fluid}
\sigmastressc_{\alpha \beta } \equiv \sigmafc \delta_{\alpha \beta}+ \eta_\fluid\left( \pderc{ \vfc^\alpha}{\beta} +\pderc{ \vfc^\beta}{\alpha}\right),
\ee
where $\epsilon_{\alpha \beta \gamma}$ is the three-dimensional Levi-Civita symbol, and $\etaf$ the two-dimensional viscosity of \ac{bflu}. Finally,   $\dint{l_\current}$ is the line element of \pomellipsec in the \ac{cur} configuration, and $\bm \hat{n}^\current$ the unit boundary normal to \pomellipsec pointing outside $\omc$.



\subsubsection{Equation of motion in the \acl{ref}-configuration coordinates}\label{sec_rigid_body_body_reference}

\bw
\begin{align}
        \label{eq_fl_rigid_rigid_reference}
        I \der{\omega}{t} & =                                               \newlinenn
        \int_\pomellipseeqr \dint{s} \, \epsilon_{\alpha \beta} \left[ \varphi^{\alpha, \, t}(\bxi{s}) - \focalpoint^\alpha \right] \sigmastressr_{\beta \gamma}(\bxi{s}) \epsilon_{\gamma \lambda} F^t_{\lambda \mu}(\bxi{s}) \der{\xi^\mu(s)}{s}
\end{align}
\ew

In \cref{eq_fl_rigid_rigid_reference}, the integral in the \ac{rhs} is over the curvilinear boundary \pomellipser, which is parametrized as $\bxr = \bxi{s}$, where $s$ is the curvilinear coordinate. Also, $\epsilon_{\alpha \beta}$ is the two-dimensional Levi-Civita symbol \cite{marchiafavaAppuntiDiGeometria2005}, and $\sigmastressr$ denotes the stress tensor in the \ac{ref} configuration:
\begin{align}
        \label{eq_def_sigma_current}
        \sigmastressr^t_{\alpha \beta}(\bxr)  \equiv & \sigmastressc^t_{\alpha \beta}({\bm \varphi}^t(\bxr)) \newlinenn
        =                                            & \sigmafr(\bxr, t) \delta_{\alpha \beta} + \etaf \Big[ G^t_{\gamma \beta}(\bxr) \pderr{\vfc^\alpha}{\gamma} + \newlinenn
                                                     & G^t_{\gamma \alpha}(\bxr) \pderr{\vfc^\beta}{\gamma}\Big].
\end{align}


\subsection{Fluid motion}\label{sec_rigid_body_fl}

In this Section, we will work out the equations of motion for \ac{bflu}.

\subsubsection{Equations of motion in the \acl{cur}-configuration coordinates}\label{sec_rigid_body_fl_cur}



The equations of motion, expressed in terms of the \ac{cur} fields $\vfc$ and $\sigmafc$, are


\begin{align}
        \label{eq_fl_rigid_fl_1_current}
        \rho_\fluid \left( \partial_t \vfc^\alpha + \vfc^\beta  \pderc{\vfc^\alpha}{\beta} \right) & =  \pderc{  \sigmafc}{\alpha} + \eta_\fluid \pder{}{\xc^\beta} \pder{\vfc^\alpha}{\xc^\beta} ,   \\
        \label{eq_fl_rigid_fl_2_current}
        \pderc{ \vfc^\alpha }{\alpha}                                                              & =                                                              0                               ,
\end{align}
and they hold for $\xc \in \omc$. \Cref{eq_fl_rigid_fl_1_current,eq_fl_rigid_fl_2_current} are the \ac{ns} and continuity equation, respectively, in a flat, two-dimensional geometry \cite{landauFluidMechanics1987}---an example of the \ac{ns} equations on curved geometry will be presented in X. There, $\rhof$ is  the two-dimensional \ac{bflu} density.


The \acp{bc} for \cref{eq_fl_rigid_fl_1_current,eq_fl_rigid_fl_2_current,eq_fl_rigid_rigid_current_1,eq_fl_rigid_rigid_current_2} are
\begin{align}
        \label{bc_fl_rigid_1_current}
        \vfc^1(\bxc)                  & =  v_\ast \; \text{ on } \pomineqc,                                                                                  \\
        \label{bc_fl_rigid_2_current}
        \vfc^2  (\bxc)                & =  0 \; \text{ on } \pomineqc ,                                                                                      \\
        \label{bc_fl_rigid_3_current}
        {\bm \vfc}    (\bxc)          & = \bzero \; \text{ on } \pomweqc ,                                                                                   \\
        \label{bc_fl_rigid_4_current}
        \vfc^\alpha (\bxc)            & = \der{[R(\theta)]_{\alpha \beta} }{t} [\psi^{t \, \beta}(\bxc) - \focalpoint^\beta] \; \text{ on } \pomellipseeqc , \\
        \label{bc_fl_rigid_5_current}
        \etaf \pderc{ \vfc^\alpha}{1} & = 0  \; \text{ on } \pomouteqc ,                                                                                     \\
        \label{bc_fl_rigid_6_current}
        \sigmafc (\bxc)               & = 0 \; \text{ on } \pomouteqc.
\end{align}
\Cref{bc_fl_rigid_1_current,bc_fl_rigid_2_current} ensure that \ac{bflu} is injected on the boundary \pominc{} in a direction parallel to the $\xc^1$ axis, and \cref{bc_fl_rigid_3_current} is a no-slip \ac{bc} \cite{landauFluidMechanics1987} along the channel walls \pomwc, see \cref{fig_ref_curr}.
\Cref{bc_fl_rigid_4_current} ensures that, at the boundary \pomellipsec,  the \ac{bflu} velocity matches the rotational velocity of \ac{bod}. Given that \ac{bod} can only rotate about its left focal point $\bfocalpoint$, the \ac{bod} rotational velocity of a point $\bxc \in \pomellipseeqc$ is given by
\be
\der{[R(\theta)]_{\alpha \beta}}{t}[\psi^{t \, \beta}(\bxc) - c_\beta],
\ee
where $\bm R$ is the  rotation matrix corresponding to a counterclockwise rotation:
\be
{\bm R}(\theta) \equiv
\left(
\begin{matrix}
                \cos \theta & - \sin \theta \\
                \sin \theta & \cos \theta
        \end{matrix}
\right).
\ee
Finally, \cref{bc_fl_rigid_5_current,bc_fl_rigid_6_current} ensure, respectively, that on \pomoutc there is zero traction and tension; this corresponds to the hypothesis that there is free flow at \pomoutc \cite{loggAutomatedSolutionDifferential2012,landauFluidMechanics1987}.

\subsubsection{Equations of motion in the \acl{ref}-configuration coordinates}\label{sec_rigid_body_fl_ref}



For each field ${\bm \vfc}({\bxc}), \sigmaf(\bxc)$ which depends on the coordinate $\bxc$ in the \ac{cur} configuration, we introduce its counterpart in the \ac{ref} configuration:

\begin{align}
        \label{eq_vf_ref}
        {\bm \vfc}(\bxr, t) & \equiv {\bm \vfr}( \bm{\varphi}^t(\bxr), t),     \\
        \label{eq_sigmaf_ref}
        \sigmafc(\bxr, t)   & \equiv {\bm \sigmafr}( \bm{\varphi}^t(\bxr), t).
\end{align}

We will now rewrite the equations of motion and the \acp{bc}   in terms of ${\bm \vfc}$, $\sigmafc$, and the \ac{ref} coordinate $\bxr$. The equations of motion \crefs{eq_fl_rigid_fl_1_current,eq_fl_rigid_fl_2_current} yield
\bw
\begin{align}
        \label{eq_fl_rigid_fl_1_reference}
        \rhof \left\{ \pder{ \vfr^\alpha(\bxr, t)}{t} + \left[ \vfr^\gamma(\bxr, t)  - \der{\varphi^{\gamma\, t}(\bxr)}{t}\right] G^t_{\beta \gamma}(\bxr) \frac{\partial  \vfr^\alpha(\bxr, t)}{\partial \xr^\beta} \right\}                                         & =              \newlinenn
        G^t_{\beta \alpha}(\bxr)  \,          \pderr{ \sigmafr(\bxr, t)   }{\beta }                + \etaf \,              G^t_{\delta \beta}(\bxr)            \,         \pderr{}{\delta} \left[  G^t_{\gamma \beta}(\bxr) \pderr{ \vfr(\bxr, t)}{\gamma}   \right], &                           \\
        \label{eq_fl_rigid_fl_2_reference}
        G^t_{\beta \alpha}(\bxr) \pderr{ \vfr^\alpha(\bxr, t)}{\beta}                                                                                                                                                                                                 & = 0,
\end{align}
\ew
which hold in $\omr$. In \cref{eq_fl_rigid_fl_1_reference,eq_fl_rigid_fl_2_reference} the inverse of the deformation-gradient tensor is
\be
\label{eq_def_G}
G^t_{\alpha \beta}(\bxr) \equiv [F^t(\bxr)]^{-1}_{\alpha \beta}.
\ee

The \acp{bc}  \crefs{bc_fl_rigid_1_current,bc_fl_rigid_2_current,bc_fl_rigid_3_current,bc_fl_rigid_4_current,bc_fl_rigid_5_current,bc_fl_rigid_6_current} are rewritten as

\begin{align}
        \label{bc_fl_rigid_1_reference}
        \vfr^1(\bxr)                                          & =  v_\ast \; \text{ on } \pomineqr,                                                                    \\
        \label{bc_fl_rigid_2_reference}
        \vfr^2      (\bxr)                                    & =  0 \; \text{ on } \pomineqr ,                                                                        \\
        \label{bc_fl_rigid_3_reference}
        {\bm \vfr} (\bxr)                                     & = {\bm 0} \; \text{ on } \pomweqr ,                                                                    \\
        \label{bc_fl_rigid_4_reference}
        \vfr^\alpha(\bxr)                                     & = \der{[R(\theta)]_{\alpha \beta} }{t} [\xr^\beta - \focalpoint^\beta] \; \text{ on } \pomellipseeqr , \\
        \label{bc_fl_rigid_5_reference}
        \etaf G^t_{\beta 1}(\bxr) \pderr{ \vfr^\alpha}{\beta} & = 0  \; \text{ on } \pomouteqr ,                                                                       \\
        \label{bc_fl_rigid_6_reference}
        \sigmafr (\bxr)                                       & = 0 \; \text{ on } \pomouteqr.
\end{align}

% Overall, \cref{\eqsfluidbodyref} depend on the \ac{ref} fields and coordinate only, and they constitute, at any time $t$, a  \ac{bvp} which, combined with the temporal \acp{bc},

% determines $\bm \vfr$ and $\sigmafr$ as functions of space and time.


\subsection{Motion of \acl{dom}}
\label{sec_eq_fluid_domain}


The equations of motion \crefs{\eqsfluidbodyref} involve the displacement field $\bm u$, which describes how the region, or domain, where \ac{bflu} sits, is deformed in time, cf. \cref{fig_ref_curr}. We will denote this region as the  \ac{dom}.
We observe that this field has, so far, been left arbitrary, and we have the freedom to choose it.

Here, $\bu$ will be determined with an analogy with the theory of elasticity \cite{kamenskyLectureNotesMAE}. We imagine that, as \ac{bod} rotates about its focal point,  each point $\bxr \in \omr$ moves into a point $\bxc \in \omc$, and that the domain $\omr$ is deformed, e.g., locally stretched and compressed, into $\omc$, like an elastic medium.

We will thus determine $\bu$ as the solution of a \ac{bvp} given by the following elastic model \cite{kamenskyLectureNotesMAE,marsdenMathematicalFoundationsElasticity1994}

\be\label{eq_mesh_fluid_rigid}
\pder{S_{\alpha \beta}(\bu)}{\xr^\beta} = 0 \text{ in } \omr,
\ee
where the elastic stress tensor $S$ is given by
\be
\label{eq_def_S}
S_{\alpha \beta} \equiv F_{\alpha \gamma} T_{\gamma \beta},
\ee
and
\begin{align}
        \label{eq_ela_1}
        T_{\alpha \beta} & \equiv \kel(\bu) E_{\gamma \gamma} \delta_{\alpha \beta} +2 \muel(\bu) \left[E_{\alpha \beta} - \frac{1}{2} \delta_{\alpha \beta} E_{\gamma \gamma}\right], \\
        \label{eq_ela_2}
        E_{\alpha \beta} & \equiv \frac{1}{2}\left( F_{\gamma \alpha}F_{\gamma \beta} - \delta_{\alpha \beta}\right),                                                                  \\
        \label{eq_ela_3}
        \kel(\bu)        & \equiv \frac{1}{[\det(F(\bu))]^\zeta},                                                                                                                      \\
        \label{eq_ela_4}
        \muel(\bu)       & \equiv \frac{1}{[\det(F(\bu))]^\zeta}.
\end{align}
In \cref{eq_ela_3,eq_ela_4}, we included an additional dependence of bulk modulus $\kel$ and the modulus of compression $\muel$ on the deformation-gradient tensor \crefs{eq_def_F}. In the absence of this dependence, the elastic model above would be unstable under local compression. In fact, the model's Lagrangian would not penalize configurations with  small $\det F$. The inclusion of the dependence \crefs{eq_ela_3,eq_ela_4} on $\det F$  allows for penalizing these configurations---the larger the exponent $\zeta > 0$, the stronger the penalty. As a result, this dependence makes the model stable under mesh compression, such as the one shown below \ac{bod} in \cref{fig_ref_curr}B.

The \acp{bc} for the  \ac{pde} \crefs{eq_mesh_fluid_rigid} are
\begin{align}
        \label{bc_ela_1}
        \bu & =\bzero \text{ on } \pomsqeqr,                                                                 \\
        \label{bc_ela_2}
        \bu & = \bfocalpoint - \bxr + {\bm R}(\theta)\cdot (\bxr - \bfocalpoint) \text{ on } \pomellipseeqr.
\end{align}
\Cref{bc_ela_1} enforces the fact that the mesh is not deformed at the boundary \pomsqr. Finally,  \cref{bc_ela_2} ensures that the mesh deformation at \pomellipser  matches the deformation induced by the rotation of \ac{bod}.

\Cref{eq_mesh_fluid_rigid,bc_ela_1,bc_ela_2} constitute the \ac{bvp} which determines $\bu$. Along the lines of \acp{bvp} in the theory of elasticity, such \ac{bvp} is already, naturally formulated in the \ac{ref} configuration \cite{landauTheoryElasticity1986}.


From  the \ac{bvp} above, we obtain a \ac{bvp} for $\dot{\bu}$, which will be needed in the following to describe the \ac{dom} motion. By deriving both sides of \cref{eq_mesh_fluid_rigid,bc_ela_1,bc_ela_2}, we obtain, respectively,
\begin{align}
        \label{eq_mesh_fluid_rigid_dot}
        \pder{}{\xr^\beta}\der{S_{\alpha \beta}(\bu)}{t} & = 0 \text{ in } \omr,                                                                  \\
        \label{bc_ela_1_dot}
        \budot                                           & =\bzero \text{ on } \pomsqeqr,                                                         \\
        \label{bc_ela_2_dot}
        \budot                                           & =  \omega \der{{\bm R}}{\theta}\cdot (\bxr - \bfocalpoint) \text{ on } \pomellipseeqr,
\end{align}
where in \cref{bc_ela_1_dot,bc_ela_2_dot} we used the fact that $\bxr$ is independent of time, and in \cref{bc_ela_2_dot} we substituted \cref{eq_fl_rigid_rigid_current_2}. The \ac{lhs} of \cref{eq_mesh_fluid_rigid_dot} depends on the second partial derivatives of $\bu$ and $\budot$---see \cref{eq_def_S,eq_ela_1,eq_ela_2,eq_ela_3,eq_ela_4}. As a result, given $\theta$, $\omega$, and $\bu$ from \cref{eq_mesh_fluid_rigid,bc_ela_1,bc_ela_2}, the \ac{bvp} \crefs{eq_mesh_fluid_rigid_dot,bc_ela_1_dot,bc_ela_2_dot}  yields $\budot$.\vspace{0.5cm}\\

Overall \cref{eq_fl_rigid_rigid_current_2,eq_fl_rigid_rigid_reference,eq_fl_rigid_fl_1_reference,eq_fl_rigid_fl_2_reference,eq_mesh_fluid_rigid,eq_mesh_fluid_rigid,bc_fl_rigid_1_reference,bc_fl_rigid_2_reference,bc_fl_rigid_3_reference,bc_fl_rigid_4_reference,bc_fl_rigid_5_reference,bc_fl_rigid_6_reference,bc_ela_1,bc_ela_2,eq_mesh_fluid_rigid_dot,bc_ela_1_dot,bc_ela_2_dot}
constitute a \ac{bvp} for $\theta, \omega, {\bm \vfr}, \sigmafr$, $\bu$ and $\budot$ which, combined with the temporal \acp{bc}
\begin{align}
        \theta(t=0)               & = \theta_0                \\
        \omega(t=0)               & = \omega_0                \\
        {\bm \vfr}(\bxr, t=0)     & = {\bm \vfr}_0(\bxr),     \\
        {\bm \sigmafr}(\bxr, t=0) & = {\bm \sigmafr}_0(\bxr),
\end{align}
determine the dynamics of the system.


\subsection{Temporal discretization}\label{sec_temp_discretization}


To numerically solve for the dynamics in a time span $0 \leq t \leq T$, we discretize time by setting $\deltat \equiv T/N$, $t_n \equiv n\, \deltat$, with $n = 0, 1, \cdots, N$.

We combine the \ac{cn} discretization method for \ac{ns} equations \cite{loggAutomatedSolutionDifferential2012,crankPracticalMethodNumerical1947} with the \ac{ale} formulation, by discretizing the dynamical equations for \ac{bod}, \ac{bflu} and \ac{dom}---see below.

\subsubsection{Rigid \acl{bod}}

Proceeding along the \ac{cn} scheme \cite{crankPracticalMethodNumerical1947}, we define fields at staggered,  integer and semi-integer time steps  scheme \cite{loggAutomatedSolutionDifferential2012}. We set
\begin{align}
        \sigmafr^n(\bxr)       & \equiv \sigmafc(\bxr, t_n)                               \\
        \sigmafr^{n-1/2}(\bxr) & \equiv \sigmafc\left(\bxr, \frac{t_n+t_{n-1}}{2}\right),
\end{align}
and similarly for other quantities.

We discretize \cref{eq_fl_rigid_rigid_reference,eq_fl_rigid_rigid_current_2} and, neglecting $\odeltat$, we obtain
\bw
\begin{align}
        \label{eq_def_omega_disc}
        \frac{\theta^n - \theta^{n-1}}{\deltat}     = & \omega^n       ,                                                                                                                                                 \\
        \label{eq_fl_rigid_rigid_reference_disc}
        \frac{\omega^n - \omega^{n-1}}{\deltat}  =    &
        I  \int_\pomellipseeqr \dint{s} \, \epsilon_{\alpha \beta} \left[ \xi^\alpha(s) + u^{n-1,\, \alpha}(\bxi{s})- \focalpoint^\alpha \right] \times                                                                            \newlinenn
                                                      & \sigmastressr^{n-1}_{\beta \gamma}(\bxi{s}; {\bm \vfr^{n-1}, \sigmafr^{n-3/2}}) \epsilon_{\gamma \lambda} F^{n-1}_{\lambda \mu}(\bxi{s}) \der{\xi^\mu(s)} {s}  ,
\end{align}
\ew
where in \cref{eq_fl_rigid_rigid_reference_disc} we used \cref{eq_def_u,eq_def_phi} and we  explicitly indicated the dependence of the stress tensor \crefs{eq_def_sigma_current} on $\bm \vfc$ and $\sigmafc$.




\subsubsection{\Acl{bflu}}

To discretize in time the \ac{bflu} problem, we will apply the \ac{cn} splitting scheme \cite{loggAutomatedSolutionDifferential2012,crankPracticalMethodNumerical1947} to the \ac{ns} equations \crefs{eq_fl_rigid_fl_1_reference,eq_fl_rigid_fl_2_reference} in the \ac{ref} configurations. This splitting scheme is derived form the \acl{ipcs} \cite{godaMultistepTechniqueImplicit1979}, in which the solution of the \ac{ns} equations is split into three intermediate steps. In what follows, we will synonymously use the terms `pressure' and `surface tension'  for conciseness \cite{worthmullerIRENEFluIdLayeR2025}.

The discrete form of the \acp{pde} \cref{eq_fl_rigid_fl_1_reference,eq_fl_rigid_fl_2_reference} is
\bw
\begin{align}
        \label{eq_fl_rigid_fl_1_reference_disc}
        \rhof \left[  \frac{\vfr^{n, \, \alpha} - \vfr^{n-1,\, \alpha}}{\deltat} + \frac{3}{2} ( \vfr^{n-1,\, \gamma}  -  \dot{u}^{n-1,\, \gamma}) \, G^{n-1}_{\beta \gamma} \frac{\partial  \vfr^{n-1, \, \alpha}}{\partial \xr^\beta} - \frac{1}{2} ( \vfr^{n-2,\, \gamma}  -  \dot{u}^{n-2,\, \gamma}) \, G^{n-2}_{\beta \gamma} \frac{\partial  \vfr^{n-2, \, \alpha}}{\partial \xr^\beta} \right] & =              \newlinenn
        G^{n-1}_{\beta \alpha}  \,          \pderr{ \sigmafr^{n-1/2}   }{\beta }                + \etaf \,              G^{n-1}_{\delta \beta}            \,         \pderr{}{\delta} \left[  G^{n-1}_{\gamma \beta} \pderr{ }{\gamma}\left(\frac{\vfr^{n, \, \alpha} + \vfr^{n-1, \, \alpha}}{2}\right)   \right],                                                                                    &                           \\
        \label{eq_fl_rigid_fl_2_reference_disc}
        G^{n-1}_{\beta \alpha} \pderr{ \vfr^{n-1,\, \alpha}}{\beta}                                                                                                                                                                                                                                                                                                                                    & = 0,
\end{align}
\ew
where we omitted the dependence on $\bxr$ for clarity. Also, in  the \ac{lhs} of \cref{eq_fl_rigid_fl_1_reference_disc} we rewrote the convective term  according to the Adams-Bashforth  discretization scheme \cite{SolvingOrdinaryDifferential1993,loggAutomatedSolutionDifferential2012}, and we used \cref{eq_def_u,eq_def_phi} to rewrite the time derivative of $\bm \varphi$ in terms of $\budot$.

The discrete form of the \acp{bc} \crefs{bc_fl_rigid_1_reference,bc_fl_rigid_2_reference,bc_fl_rigid_3_reference,bc_fl_rigid_4_reference,bc_fl_rigid_5_reference,bc_fl_rigid_6_reference} reads
\bw
\begin{align}
        \label{bc_fl_rigid_1_reference_disc}
        \vfr^{n, \, 1}                                                                                                & = v_\ast \; \text{ on }\pomineqr,                                                                                                         \\
        \label{bc_fl_rigid_2_reference_disc}
        \vfr^{n, \, 2}                                                                                                & =  0 \; \text{ on } \pomineqr ,                                                                                                           \\
        \label{bc_fl_rigid_3_reference_disc}
        {\bm \vfr}^n                                                                                                  & = {\bm 0} \; \text{ on } \pomweqr ,                                                                                                       \\
        \label{bc_fl_rigid_4_reference_disc}
        {\bm \vfr}                                                                                                    & =
        \omega^n  \left. \der{{\bm R}}{\theta} \right|_{\theta = \theta_n} \cdot (\bxr - \bfocalpoint)  \text{ on } \pomellipseeqr ,                                                                                                                              \\
        \label{bc_fl_rigid_5_reference_disc}
        \etaf \, G^{n-1}_{\beta 1} \pderr{}{\beta}\left( \frac{\vfr^{n, \, \alpha} + \vfr^{n-1, \, \alpha}}{2}\right) & = 0                                                                                                           \; \text{ on } \pomouteqr , \\
        \label{bc_fl_rigid_6_reference_disc}
        \sigmafr^{n-1/2}                                                                                              & = 0 \; \text{ on } \pomouteqr.
\end{align}
\ew

We will now split the \ac{bvp} into three steps \cite{loggAutomatedSolutionDifferential2012,chorinNumericalSolutionNavierStokes1968}:
\begin{enumerate}
        \titleditem{Approximated velocity}{
                \label{item_split_1}
                We introduce the velocity $\bvbar$, which approximates the true velocity $\bm \vfr$, and which  satisfies the \ac{bvp}
                \bw
                \begin{align}
                        \label{eq_fl_rigid_fl_1_reference_disc_aux}
                        \rhof \left\{  \frac{\vbar^\alpha - \vfr^{n-1,\, \alpha}}{\deltat} + \left[ \frac{3}{2} ( \vfr^{n-1,\, \gamma}  -  \dot{u}^{n-1,\, \gamma}) \, G^{n-1}_{\beta \gamma}  - \frac{1}{2} ( \vfr^{n-2,\, \gamma}  -  \dot{u}^{n-2,\, \gamma}) \, G^{n-2}_{\beta \gamma}  \right] \frac{\partial  V^\alpha}{\partial \xr^\beta}\right\} & =              \newlinenn
                        G^{n-1}_{\beta \alpha}  \,          \pderr{ \sigmafr^\ast   }{\beta }                + \etaf \,              G^{n-1}_{\delta \beta}            \,         \pderr{}{\delta} \left(  G^{n-1}_{\gamma \beta} \pderr{V^\alpha }{\gamma}    \right),
                \end{align}
                \begin{align}
                        \label{bc_fl_rigid_1_reference_disc_aux}
                        \vbar^1                                            & = v_\ast \; \text{ on }\pomineqr,                                                                                                        \\
                        \label{bc_fl_rigid_2_reference_disc_aux}
                        \vbar^2                                            & =  0 \; \text{ on } \pomineqr ,                                                                                                          \\
                        \label{bc_fl_rigid_3_reference_disc_aux}
                        \bvbar                                             & = {\bm 0} \; \text{ on } \pomweqr ,                                                                                                      \\
                        \label{bc_fl_rigid_4_reference_disc_aux}
                        \bvbar                                             & =
                        \omega^n  \left. \der{{\bm R}}{\theta} \right|_{\theta = \theta_n} \cdot (\bxr - \bfocalpoint)  \text{ on } \pomellipseeqr ,                                                                  \\
                        \label{bc_fl_rigid_5_reference_disc_aux}
                        \etaf \, G^{n-1}_{\beta 1} \pderr{V^\alpha}{\beta} & = 0                                                                                                           \; \text{ on } \pomouteqr.
                \end{align}
                \ew
                where
                \begin{align}
                        {\bm V}       & \equiv\frac{ \bvbar + {\bm \vfr}^{n-1}}{2}, \\
                        \label{eq_def_sigmarast}
                        \sigmafr^\ast & \equiv \sigmafr^{n-3/2}.
                \end{align}
        }
        \titleditem{Pressure correction}{
                \label{item_split_2}

                Subtracting \cref{eq_fl_rigid_fl_1_reference_disc,eq_fl_rigid_fl_1_reference_disc_aux} and neglecting $\odeltat$, we obtain
                \be
                \label{eq_phi_1}
                \rhof \frac{\vbar^\alpha - \vfr^{n, \, \alpha}}{\deltat}  = G^{n-1}_{\beta \alpha}  \,          \pderr{ \phi   }{\beta },
                \ee
                where
                \be\label{eq_def_phi_pressure}
                \phi \equiv \sigmafc^\ast - \sigmafc^{n-1/2}
                \ee
                is the surface-tension increment. \Cref{bc_fl_rigid_6_reference_disc,eq_def_sigmarast,eq_def_phi_pressure} imply the \ac{bc}
                \be
                \label{bc_phi_1}
                \phi = 0, \, \text{ on } \pomouteqr.
                \ee
                Multiplying \cref{eq_phi_1} by $G^{n-1}_{\gamma \alpha} \hat{n}^{\reference\, \gamma}$, where $\hat{\bm n}^{\reference}$ is the boundary normal in the \ac{ref} configuration,  we obtain
                \begin{align}
                        \label{eq_phi_2}
                        \frac{\rhof}{\deltat} G^{n-1}_{\gamma \alpha} \hat{n}^{\reference\, \gamma} (\vbar^\alpha - \vfr^{n, \, \alpha}) & = \newlinenn
                        \hat{n}^{\reference\, \gamma} G^{n-1}_{\gamma \alpha}  G^{n-1}_{\beta \alpha}  \,          \pderr{ \phi   }{\beta }.
                \end{align}
                Combining \cref{eq_phi_2} with \cref{bc_fl_rigid_1_reference_disc,bc_fl_rigid_2_reference_disc,bc_fl_rigid_3_reference_disc,bc_fl_rigid_4_reference_disc,bc_fl_rigid_1_reference_disc_aux,bc_fl_rigid_2_reference_disc_aux,bc_fl_rigid_3_reference_disc_aux,bc_fl_rigid_4_reference_disc_aux}, we obtain the second \ac{bc}
                \begin{align}
                        \label{bc_phi_2}
                        \hat{n}^{\reference\, \gamma} G^{n-1}_{\gamma \alpha}  G^{n-1}_{\beta \alpha}  \,          \pderr{ \phi   }{\beta } & = \newlinenn 0
                        \text{ on } \pomineqc \cup \pomweqc \cup \pomellipseeqc                                                             & .
                \end{align}
                The \ac{bvp} \crefs{eq_phi_2,bc_phi_1,bc_phi_2} determines $\phi$ and, through the definition \crefs{eq_def_phi_pressure}, the surface tension $\sigmac^{n-1/2}$.
        }
        \titleditem{Velocity correction}{
                \label{item_split_3}
                The velocity ${\bm \vfr}^n$ is determined from the approximate velocity $\bvbar$ from \cref{eq_phi_1}.

        }
\end{enumerate}



%sign

\subsubsection{\Acl{dom}}

The discrete version of  the \ac{bvp} for $\bu$, \cref{eq_mesh_fluid_rigid,bc_ela_1,bc_ela_2}, reads
\begin{align}
        \label{eq_mesh_fluid_rigid_disc}
        \pder{S_{\alpha \beta}(\bu^n)}{\xr^\beta} & = 0 \text{ in } \omr,                                                                            \\
        \label{bc_ela_1_disc}
        \bu^n                                     & =\bzero \text{ on } \pomsqeqr,                                                                   \\
        \label{bc_ela_2_disc}
        \bu^n                                     & = \bfocalpoint - \bxr + {\bm R}(\theta^n)\cdot (\bxr - \bfocalpoint) \text{ on } \pomellipseeqr,
\end{align}
and that of the \ac{bvp} for $\budot$, \cref{eq_mesh_fluid_rigid_dot,bc_ela_1_dot,bc_ela_2_dot}, is
\begin{align}
        \label{eq_mesh_fluid_rigid_disc_dot}
        \pder{}{\xr^\beta}\der{S_{\alpha \beta}(\budot^n)}{t} & = 0 \text{ in } \omr,                                                                  \\
        \label{bc_ela_1_disc_dot}
        \budot  ^n                                            & =\bzero \text{ on } \pomsqeqr,                                                         \\
        \label{bc_ela_2_disc_dot}
        \budot  ^n                                            & =  \omega \der{{\bm R}}{\theta}\cdot (\bxr - \bfocalpoint) \text{ on } \pomellipseeqr,
\end{align}\vspace{0.5cm}\\

We can now iterate in time for the ensemble of fields as follows. At each step $n$, given $\theta^{n-1}, \omega^{n-1}, {\bm \vfr}^{n-1}, \sigmafr^{n-3/2}$ $\bu^{n-1}$ and $\budot^{n-1}$ from the preceding step, we
\begin{enumerate}
        \titleditem{Update \ac{bod}}{ \label{fl_rigid_step_1}
                Solve the algebraic equations \cref{eq_def_omega_disc,eq_fl_rigid_rigid_reference_disc} for $\theta^n$ and $\omega^n$.
        }
        \titleditem{Update \ac{dom}}{ \label{fl_rigid_step_2}
                Solve  the \acp{bvp} \crefs{eq_mesh_fluid_rigid_disc,bc_ela_1_disc,bc_ela_2_disc,eq_mesh_fluid_rigid_disc_dot,bc_ela_1_disc_dot,bc_ela_2_disc_dot} for $\bu^n$ and $\budot^n$.
        }
        \titleditem{Update \ac{bflu}}{
                \label{fl_rigid_step_3}
                Solve the \acp{bvp} \crefs{eq_fl_rigid_fl_1_reference_disc_aux,bc_fl_rigid_1_reference_disc_aux,bc_fl_rigid_2_reference_disc_aux,bc_fl_rigid_3_reference_disc_aux,bc_fl_rigid_4_reference_disc_aux,bc_fl_rigid_5_reference_disc_aux} and \crefs{eq_phi_2,bc_phi_1,bc_phi_2}  for $\bvbar$ and $\sigmafr^{n-1/2}$, and obtain $\bm \vfr^n$ from \cref{eq_phi_1}.
        }
\end{enumerate}

%sign



\begin{figure*}
        \centering
        \includegraphics[width=\textwidth]{figures/figure_1/figure_1.pdf}
        \caption{
                \label{fig_fluid_rigid}
                Interaction between a  \acl{bflu} and a rigid \acl{bod}. \plab{A} Temporal snapshot of the mesh (black lines) and  velocity field (arrows) at the \ac{cur} time $t$, shown on top. The direction of the velocity field is indicated by the arrows, and its norm by their color. The  \acl{bod} (ellipse) is allowed to pivot about its left focal point (red dot), and the related angle with respect to the $\xc^1$ direction is denoted by a red arc.   \plab{B} Color map of the surface tension at the \ac{cur} instant of time $t$.
        }
\end{figure*}

