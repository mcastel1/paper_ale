\begin{figure*}
    \centering
    \includegraphics[width=\textwidth]{figures/figure_4/figure_4.pdf}
    \caption{
        \label{fig_mem}
        Helfrich fluid layer described with the generalized \acl{al} gauge. \textbf{A}) Displacement field $\vec{U}$, which relates the reference and the current configuration, shown in red and green, respectively. \textbf{B}) Tangential velocity. Arrows show the velocity direction, and the  color code the velocity norm. \textbf{C}) Normal velocity, whose value is shown with the color code.\textbf{D}) Surface tension. All panels refer to the same instant of time, shown on top.
    }
\end{figure*}




\begin{figure*}
    \centering
    \includegraphics[width=\textwidth]{figures/figure_1/figure_1.pdf}
    \caption{
        \label{fig_fluid_rigid}
        Interaction between a bulk fluid and a rigid body, on macroscopic scales. \textbf{A}) Temporal snapshot of the mesh (black lines) and velocity field (arrows). The direction of the velocity field is indicated by the arrows, and its norm by their color. The rigid body (ellipse) is allowed to pivot about its left focal point (red dot), and the related angle with respect to the $x$ direction is denoted by the red arc.   \textbf{B}) Color map of the surface tension. Both panels refer to the same instant of time, shown on top.
    }
\end{figure*}

\begin{figure*}
    \centering
    \includegraphics[width=\textwidth]{figures/figure_3/figure_3.pdf}
    \caption{
        \label{fig_mem_ela}
        Interaction between a bulk fluid and an elastic body, on macroscopic scales. The notation is the same as in \cref{fig_fluid_rigid}, and the elastic body is depicted as a red mesh.
    }
\end{figure*}