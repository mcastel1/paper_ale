\section{Fluid and Helfrich membrane}\label{sec_fl_mem}

In this Section, we will build on the analysis of \cref{sec_fl_elastic}, and complexify the structure of the elastic body with which \ac{bflu} interacts by considering, instead of an elastic material, an \ac{mem} \cite{helfrichElasticPropertiesLipid1973,derenyiFormationInteractionMembrane2002}.

Both the physical nature and mathematical description of \ac{mem} is significantly more complex than that of \ac{ela}. First, unlike \ac{ela}, \ac{mem} has a fluid structure: its material elements can flow both tangentially and normally to it.
Such fluid behavior requires an Eulerian description, which then needs to be connected to the Lagrangian description used to describe \ac{dom}. Second, \ac{mem} is described by fourth-order \acp{pde} \cite{zhong-canBendingEnergyVesicle1989}, which are significantly more complex than the second-order \acp{pde} which describe \ac{ela} \cite{worthmullerIRENEFluIdLayeR2025}. Third, the geometrical description of \ac{mem} requires fixing an appropriate \textit{gauge}.
In fact, a general parametrization of \ac{mem} is given by two functions $\bxmemc(\xcoord^1)$, both of which depend on the curvilinear coordinate $\xcoord^1$. However, a single \ac{pde}  determines the \ac{mem} shape \cite{zhong-canBendingEnergyVesicle1989}.
This redundancy is due to the fact that parametrizations with different values of the stretching
\be\label{eq_intro_nu}
\pder{{\bxmemc}}{\xcoord^1} \cdot \pder{{\bxmemc}}{\xcoord^1} \equiv \nu^2,
\ee
yield the same shape. The redundancy above may be removed by a gauge fixing, i.e., by adding a constraint on $\bxmemc$, or its derivatives.
Two common gauge choices are the Monge  \cite{hsiungFirstCourseDifferential1981,desernoFluidLipidMembranes2015}, or the arc-length parametrization \cite{derenyiFormationInteractionMembrane2002}, in which one sets
\be
\label{eq_arc_length_gauge}
\nu = 1.
\ee
The arc-length gauge has been used to describe, for instance, the steady state for membrane tubules \cite{derenyiFormationInteractionMembrane2002}. However, while a static constraint such as \crefs{eq_arc_length_gauge} is suited to  static shapes, it cannot describe the  \ac{mem} dynamics, such as the one of \ac{mem} interacting with \ac{bflu}. In fact, as \ac{mem} is deformed, it will also stretch: as a result, \cref{eq_arc_length_gauge} must be replaced by a time-dependent  constraint---see below.

We describe \ac{mem} with the Lagrangian approach, along the lines of the description of \ac{dom} in \cref{sec_fl_rigid} and \ac{ela}  in \cref{sec_fl_elastic}.  The \ac{mem} displacement field is
\be
\bumem(\xcoord^1) \equiv \bxmemc(\xcoord^1) - \bxmemr(\xcoord^1),
\ee
where we choose the \ac{ref} configuration as a straight line
\be
\bxmemr(\xcoord^1) \equiv (\xcoord^1, h), \, 0 \leq  \xcoord^1 \leq L,
\ee
which is shown in \cref{fig_ref_cur_mem}A.
%sign

The dynamical equations for \ac{mem} are

\bw
\begin{align}
    \label{eq-dyn-continuity}
    \nab_i v^i - 2 H w                                                                                                                                                                                                                                             & =                                                                                                 0, \\ \nn
    \label{eq-dyn-v}
    \rho ( \partial_t v^i + v^j \nab_j v^i  - 2 v^j w b^i_j - w \nab^i w )                                                                                                                                                                                         & =                                                                                                    \\
    \nab^i \sigma + \eta \left[ - \nablb v^i - 2 \left( b^{ij} - 2 \, H \, g^{ij}  \right) \nab_j w + 2 K v^i   \right]   ,                                                                                                                                        &                                                                                                      \\ \nn
    \label{eq-dyn-w}
    \rho \left[ \partial_t w + v^i \left( v^j b_{ji} + \nabla_i w  \right) \right]                                                                                                                                                                                 & =                                                                                                    \\
    - 2\kap \left[   \nab_i \nab^i H  + 2 H (H^2 - K)  \right] + 2 \sigma H    +                                                         2 \eta \left[ (\nab^i v^j)b_{ij} - 2 w (2 H^2 - K)                                                                \right] & ,
\end{align}
\ew

\begin{figure}
    \centering
    \includegraphics[width=\columnwidth]{figures/figure_9/figure_9.pdf}
    \caption{
        \label{fig_ref_cur_mem}
        \Acl{ref} and \acl{cur} configuration for the interaction between a  \acf{bflu} and a \acf{mem}. The figure follows the same notation as \cref{fig_ref_curr_rigid}.
        The top edge of the \ac{bflu} mesh, $\pomtopeqr$, coincides with the one-dimensional  manifold $\omlineeqr$ of \ac{mem}. In panels \textbf{A} and \textbf{B}, the domains $\omr$ and $\omc$ are the regions covered by the mesh, respectively.
    }
\end{figure}

\begin{figure*}
    \centering
    \includegraphics[width=\textwidth]{figures/figure_4/figure_4.pdf}
    \caption{
        \label{fig_mem}
        Helfrich fluid layer described with the generalized \acl{al} gauge. The layer is subjected to a gravitational field directed along the $X^2$ axis, in the direction of negative $X^2$.  \plab{A} Displacement field $\vec{U}$, which relates the reference and the current configuration, shown in red and green, respectively. \plab{B} Stretching $\nu$.  \plab{C} Tangent angle $\psi$. \plab{D} Tangential velocity. Arrows show the velocity direction, and the  color code the velocity norm. \plab{E} Normal velocity, whose value is shown with the color code. \plab{F} Surface tension. All panels refer to the same instant of time, shown on top.
    }
\end{figure*}








\begin{figure*}
    \centering
    \includegraphics[width=\textwidth]{figures/figure_5/figure_5.pdf}
    \caption{
        Interaction between a bulk fluid and a Helfrich membrane.
        \plab{A} Membrane reference and current configuration (red and green curve, respectively), and displacement field (black arrows).
        \plab{B} Membrane stretch field.
        \plab{C} Membrane tangent angle.
        \plab{D} and \plab{E}) Bulk-fluid velocity and tension; the notation is the same as in \cref{fig_fluid_rigid}.
        \plab{F}, \plab{G} and \plab{H}: membrane tangential velocity, normal velocity and tension; the notation is the same as in \cref{fig_mem}B, C and D, respectively.  All panels refer to the same instant of time, shown on top, and display also the deformed mesh (gray lines).
        \label{fig_fluid_mem}
    }
\end{figure*}
