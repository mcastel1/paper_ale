\begin{enumerate}
    \item \cite{derenyiFormationInteractionMembrane2002} study the steady state with arc-length gauge
    \item In the dynamics, arc-length gauge does not work
    \item We show how to use a generalized arc-length gauge and how to correctly incorporate it into the dynamics to obtain a proper dynamics, \cref{fig_mem}.
    \item In this dynamics
          \begin{itemize}
              \item we implement the Crank-Nicolson method to the equations written in the reference configuration, and prove it to be numerically stable
              \item we show that the derivative of mesh motion $\dot{\bu}$ can be obtained exactly from an auxiliary PDE withotu recurring to error-prone time discretization.
          \end{itemize}
    \item We couple the membrane dynamics to a bulk fluid. The membrane is a complex elastic body -> We proceed by steps:
          \begin{enumerate}
              \item Bulk fluid + rigid body \cref{fig_fluid_rigid}
              \item Elastic body with stable elastic model (stable under compression)
              \item Bulk fluid + elastic body with stable elastic model, \cref{fig_fluid_ela}
              \item Bulk fluid + membrane, \cref{fig_fluid_mem}
                    \begin{itemize}
                        \item Bulk fluid and membrane both described with Crank-Nicolson method -> we prove that the resulting dynamics is numerically stable. Say that you found a way to teplace $t \rightarrow t_n$ that is stable and that you could have used other ways which are also correct to within $\odeltat$.
                        \item It allows for overhangs
                        \item It is the first study that describes a bulk fluid coupled to a Helfrich membrane with ALE
                        \item It allows for turbulent behavior for both the bulk fluid and the membrane tangential flow
                    \end{itemize}
          \end{enumerate}
\end{enumerate}

Perspectives:
\begin{enumerate}
    \item Study nucleoid compaction in E. Coli with ALE
\end{enumerate}