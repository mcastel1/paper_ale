\section{Fluid and elastic body}\label{sec_fl_elastic}

In this Section, we will build on the analysis of \cref{sec_fl_rigid} and put \ac{bflu} into interaction with a more complex physical object: an \ac{ela}. Only the main results will be presented here; their derivation follows the lines of  \cref{sec_fl_rigid}.

The \ac{ref} and \ac{cur} configurations are shown in \cref{fig_ref_curr_ela}. Here, the \ac{ela} boundary $\pomcircouteq$ may deform, but the inner \ac{ela} boundary $\pomcircineq$ is pinned to the $x^1$, $x^2$ plane.



\begin{figure*}
        \centering
        \includegraphics[width=\textwidth]{figures/figure_7/figure_7.pdf}
        \caption{
                \label{fig_ref_curr_ela}
                \Acl{ref} and \acl{cur} configuration for the interaction between a  \acl{bflu} and an \acl{ela}. The figures follows the same notation as \cref{fig_ref_curr_rigid}.
        }
\end{figure*}


The equations of motion are the following. First, we have the \ac{ns} and mass-conservation equations \crefs{eq_fl_rigid_fl_1_current,eq_fl_rigid_fl_2_current}, respectively. Second,  the motion of \ac{ela} is governed by the elasticity equations
\be
\rhoela \dern{\buela^\alpha(\bxr)}{t}{2} = \pderr{\Sela_{\alpha \beta}(\buela)}{\beta},
\ee
where $\buela$ and  $\rhoela$ are, respectively, the deformation field and density  of \ac{ela}; the latter is  assumed to be independent of space. The \ac{ela} stress tensor is derived from a compressible neo-Hookean model \cite{bazilevsIsogeometricFluidstructureInteraction2008}, and it reads
\be
\Sela_{\alpha \beta} \equiv \frac{\mu}{\det(F)}\left(- \frac{1}{2} C_{\gamma \gamma} \right)
\ee

%sign


\begin{figure*}
        \centering
        \includegraphics[width=\textwidth]{figures/figure_3/figure_3.pdf}
        \caption{
                \label{fig_fluid_ela}
                Interaction between a \acl{bflu} and an \acl{ela}. The notation is the same as in \cref{fig_fluid_rigid}, and the \acl{ela} is depicted as a red mesh.
        }
\end{figure*}