\section{Fluid and elastic body}\label{sec_fl_elastic}

In this Section, we will build on the analysis of \cref{sec_fl_rigid} and put \ac{bflu} into interaction with a more complex physical object: an \ac{ela}. Only the main results will be presented here; their derivation follows the lines of  \cref{sec_fl_rigid}.

The \ac{ref} and \ac{cur} configurations are shown in \cref{fig_ref_curr_ela}. Here, the \ac{ela} boundary $\pomcircouteq$ may deform, but the inner \ac{ela} boundary $\pomcircineq$ is pinned to the $x^1$, $x^2$ plane.



\begin{figure*}
        \centering
        \includegraphics[width=\textwidth]{figures/figure_7/figure_7.pdf}
        \caption{
                \label{fig_ref_curr_ela}
                \Acl{ref} and \acl{cur} configuration for the interaction between a  \acl{bflu} and an \acl{ela}. The figures follows the same notation as \cref{fig_ref_curr_rigid}.
        }
\end{figure*}


The equations of motion are the following.

First, the dynamics of \ac{bflu} is governed by the  \ac{ns} and mass-conservation equations \crefs{eq_fl_rigid_fl_1_current,eq_fl_rigid_fl_2_current}, respectively. Their \acp{bc} are \cref{bc_fl_rigid_1_current,bc_fl_rigid_2_current,bc_fl_rigid_3_current,bc_fl_rigid_5_current,bc_fl_rigid_6_current}, which stay unchanged. On the other hand, the \ac{bc} \crefs{bc_fl_rigid_4_current}, which ensures that the \ac{bflu} velocity matches the \ac{bod} velocity at the interface between \ac{bflu} and \ac{bod}, now reads
\be
\label{bc_fl_ela_4_current}
\bvfc(\bxc) = \left. \der{\buela(\bxr)}{t} \right|_{\bxr = \psi^t(\bxc)} \text{ on } . \pomcircouteqc
\ee


Second,  the motion of \ac{ela} is governed by Newton's equations of motion for an elastic body \cite{landauTheoryElasticity1986}
\be\label{eq_fl_elastic_ela_current}
\rhoela \dern{\buela^\alpha(\bxr)}{t}{2} = \pderr{\Sela_{\alpha \beta}(\buela)}{\beta},
\ee
where $\buela$ and  $\rhoela$ are, respectively, the deformation field and density  of \ac{ela}; the latter is  assumed to be independent of space. The \ac{ela} stress tensor is derived from a compressible neo-Hookean model \cite{bazilevsIsogeometricFluidstructureInteraction2008}, and it reads
\begin{align}
        \label{eq_def_S_neo_hookean}
        \Sela_{\alpha \beta} \equiv \frac{\muel}{\det(F)}\left(- \frac{1}{2} C_{\gamma \gamma} G_{\alpha \beta} + F_{\beta \alpha}\right) + & \newlinenn
        \kel [(\det(F))^2-1] G_{\alpha \beta},
\end{align}
where
\be
\label{eq_def_C}
C_{\alpha \beta } \equiv \delta_{\alpha \beta} + 2 E_{\alpha \beta}.
\ee
We have chosen this model because it is stable under compression, i.e., its potential energy diverges as $\det(F) \rightarrow 0$ \cite{bazilevsIsogeometricFluidstructureInteraction2008}. As opposed to linear models \cite{landauTheoryElasticity1986} and to the elastic model of  \cref{sec_eq_fluid_domain}, this model  proves to yield a stable dynamics for \ac{ela} as it is compressed  by  \ac{bflu}, see below and \cref{fig_fluid_ela}.
The \acp{bc} for \cref{eq_fl_elastic_ela_current} are
\begin{align}
        \label{bc_fl_ela_1_current}
        \buela(\bxr)                                                                                                              & =0               \text{ on }\pomcircineqr,   \\
        \label{bc_fl_ela_2_current}
        \epsilon_{\beta \gamma} \, \Sela_{\alpha \beta}(\buela^t(\bxi{s})) \der{\xi^\gamma}{s}\                                   & =    \newlinenn
        \sigmastressc_{\alpha \beta}(\bm{\varphi}^t(\bxi{s})) \epsilon_{\beta \gamma} \der{{\varphi}^{t, \, \gamma} (\bxi{s})}{s} & \text{ on } \pomcircouteqr.                &
\end{align}


Finally, the equations of motion for \ac{dom} are \crefs{eq_mesh_fluid_rigid,eq_mesh_fluid_rigid_dot}, with \acp{bc} \crefs{bc_ela_1,bc_ela_1_dot} and
\begin{align}
        \label{bc_ela_3}
        \budom^t(\bxr)    & = \buela^t(\bxr) \text{ on } \pomcircouteqr,    \\
        \label{bc_ela_3_dot}
        \budomdot^t(\bxr) & = \bueladot^t(\bxr) \text{ on } \pomcircouteqr.
\end{align}
\Cref{bc_ela_3} and \cref{bc_ela_3_dot} ensure that the displacement and velocity of \ac{ela} and  \ac{dom} are conformal at the \ac{ela}-\ac{dom} interface, and they replace  \cref{bc_ela_2,bc_ela_2_dot}, respectively.


%sign

\begin{figure*}
        \centering
        \includegraphics[width=\textwidth]{figures/figure_3/figure_3.pdf}
        \caption{
                \label{fig_fluid_ela}
                Interaction between a \acl{bflu} and an \acl{ela}. The notation is the same as in \cref{fig_fluid_rigid}, and the \acl{ela} is depicted as a red mesh.
        }
\end{figure*}