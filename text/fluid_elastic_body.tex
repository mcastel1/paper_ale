\section{Fluid and elastic body}\label{sec_fl_elastic}

In this Section, we will build on the analysis of \cref{sec_fl_rigid} and put \ac{bflu} into interaction with a more complex physical object: an \ac{ela}. Only the main results will be presented here; their derivation follows the lines of  \cref{sec_fl_rigid}.

The \ac{ref} and \ac{cur} configurations are shown in \cref{fig_ref_curr_ela}. Here, the \ac{ela} boundary $\pomcircouteq$ may deform, but the inner \ac{ela} boundary $\pomcircineq$ is pinned to the $\xc^1$, $\xc^2$ plane.



\begin{figure*}
        \centering
        \includegraphics[width=\textwidth]{figures/figure_7/figure_7.pdf}
        \caption{
                \label{fig_ref_curr_ela}
                \Acl{ref} and \acl{cur} configuration for the interaction between a  \acl{bflu} and an \acl{ela}. The figures follows the same notation as \cref{fig_ref_curr_rigid}. In panels \textbf{A} and \textbf{B}, the domains $\omr$ and $\omc$ are the regions covered by the mesh, respectively.
        }
\end{figure*}


The equations of motion are the following.

First, the dynamics of \ac{bflu} is governed by the  \ac{ns} and mass-conservation equations \crefs{eq_fl_rigid_fl_1_current,eq_fl_rigid_fl_2_current}, respectively. Their \acp{bc} are \cref{bc_fl_rigid_1_current,bc_fl_rigid_2_current,bc_fl_rigid_3_current,bc_fl_rigid_5_current,bc_fl_rigid_6_current}, which stay unchanged. On the other hand,  \ac{bc} \crefs{bc_fl_rigid_4_current}, which ensures that the \ac{bflu} velocity matches the \ac{bod} velocity at the interface between \ac{bflu} and \ac{bod}, now reads
\be
\label{bc_fl_ela_4_current}
\bvfc(\bxc) = \left. \der{\buela(\bxr)}{t} \right|_{\bxr = \psi^t(\bxc)} \text{ on }  \pomcircouteqc
\ee


Second,  the motion of \ac{ela} is governed by Newton's equations of motion for an elastic body \cite{landauTheoryElasticity1986}
\be\label{eq_fl_elastic_ela_current}
\rhoela \dern{\uela^\alpha(\bxr)}{t}{2} = \pderr{\Sela_{\alpha \beta}(\buela)}{\beta},
\ee
where  $\rhoela$ is the density  of \ac{ela} which is assumed to be independent of space.
Also, $\buela$ is the deformation field of \ac{ela}, which is defined along the lines of \cref{eq_def_u}.
The \ac{ela} stress tensor is derived from a compressible neo-Hookean model \cite{bazilevsIsogeometricFluidstructureInteraction2008}, and it reads
\begin{align}
        \label{eq_def_S_neo_hookean}
        \Sela_{\alpha \beta} \equiv \frac{\muel}{\det(F)}\left(- \frac{1}{2} C_{\gamma \gamma} G_{\alpha \beta} + F_{\beta \alpha}\right) + & \newlinenn
        \kel [(\det(F))^2-1] G_{\alpha \beta},
\end{align}
where
\be
\label{eq_def_C}
C_{\alpha \beta } \equiv \delta_{\alpha \beta} + 2 E_{\alpha \beta}.
\ee
We have chosen this model because it is stable under compression, i.e., its potential energy diverges as $\det(F) \rightarrow 0$ \cite{bazilevsIsogeometricFluidstructureInteraction2008}. As opposed to linear models \cite{landauTheoryElasticity1986} and to the elastic model of  \cref{sec_eq_fluid_domain}, this model  proves to yield a stable dynamics for \ac{ela} as it is compressed  by  \ac{bflu}, see below and \cref{fig_fluid_ela}.
The \acp{bc} for \cref{eq_fl_elastic_ela_current} are
\begin{align}
        \label{bc_fl_ela_1_current}
        \buela(\bxr)                                                                                                              & =0               \text{ on }\pomcircineqr,   \\
        \label{bc_fl_ela_2_current}
        \epsilon_{\beta \gamma} \left. \Sela_{\alpha \beta}(\buela^t)\right|_{\bxr = \bxi{s}}\der{\xi^\gamma}{s}\                 & =    \newlinenn
        \sigmastressc_{\alpha \beta}(\bm{\varphi}^t(\bxi{s})) \epsilon_{\beta \gamma} \der{{\varphi}^{t, \, \gamma} (\bxi{s})}{s} & \text{ on } \pomcircouteqr.                &
\end{align}


Finally, the equations of motion for \ac{dom} are \crefs{eq_mesh_fluid_rigid,eq_mesh_fluid_rigid_dot}, with \acp{bc} \crefs{bc_ela_1,bc_ela_1_dot} and
\begin{align}
        \label{bc_ela_3}
        \budom^t(\bxr)    & = \buela^t(\bxr) \text{ on } \pomcircouteqr,    \\
        \label{bc_ela_3_dot}
        \budomdot^t(\bxr) & = \bueladot^t(\bxr) \text{ on } \pomcircouteqr.
\end{align}
\Cref{bc_ela_3} and \cref{bc_ela_3_dot} ensure that the displacement and velocity of \ac{ela} and  \ac{dom} are conformal at the \ac{ela}-\ac{dom} interface, and they replace  \cref{bc_ela_2,bc_ela_2_dot}, respectively.



The system dynamics is obtained as follows: given $\bvfr^{n-1}$, $\sigmafr^{n-3/2}$, $\buela^{n-1}$, $\bueladot^{n-1}$, $\budomdot^{n-1}$ and $\budomdot^{n-1}$  from the preceding step, we
\begin{itemize}
        \titleditem{Update \ac{ela}}{ \label{fl_ela_step_1}
                Solve for $\buela^n$ and $\bueladot^n$ the discrete version of \cref{eq_fl_elastic_ela_current}
                \begin{align}
                        \label{eq_fl_elastic_ela_current_disc_1}
                        \rhoela \frac{\ueladot^{n, \, \alpha} -  \ueladot^{n-1, \, \alpha}}{\deltat} & = \pderr{\Sela_{\alpha \beta}(\buela^n)}{\beta}, \\
                        \label{eq_fl_elastic_ela_current_disc_2}
                        \frac{\buela^n -  \buela^{n-1}}{\deltat}                                     & = \bueladot,
                \end{align}
                with \acp{bc} obtained from \cref{bc_fl_ela_1_current,bc_fl_ela_2_current}:
                \bw
                \begin{align}
                        \label{bc_fl_ela_1_current_disc}
                        \buela^n                                                                                                          = & 0               \text{ on }\pomcircineqr, \\
                        \label{bc_fl_ela_2_current_disc}
                        \epsilon_{\beta \gamma} \, \left. \Sela_{\alpha \beta}(\buela^n)\right|_{\bxr = \bxi{s}}\der{\xi^\gamma}{s}\   =    &
                        \sigmastressr_{\alpha \beta}(\bxi{s};\bvfr^{n-1}, \sigmafr^{n-3/2}, \bu^n) \epsilon_{\beta \gamma} \left. F_{\gamma \delta}(\bu_\ela^{n-1})\right|_{\bxr = \bxi{s}}\der{ \xi^\delta}{s} \text{ on } \pomcircouteqr.
                \end{align}
                \ew
                In \cref{bc_fl_ela_2_current_disc}, we have used \cref{eq_def_phi,eq_def_F} to rewrite the derivative of ${\bm \phi}(\bxi{s})$.



        }
        \titleditem{Update \ac{dom}}{ \label{fl_ela_step_2}
                Solve for $\budom^n$ and $\budomdot^n$ the \acp{bvp} given by \cref{eq_domain_fluid_rigid_disc,eq_domain_fluid_rigid_disc_dot}  with \acp{bc} \crefs{bc_ela_1_disc}, \crefs{bc_ela_1_disc_dot} and
                \begin{align}
                        \label{bc_ela_3_disc}
                        \budom^n    & = \buela^n \text{ on } \pomcircouteqr,    \\
                        \label{bc_ela_3_dot_disc}
                        \budomdot^n & = \bueladot^n \text{ on } \pomcircouteqr.
                \end{align}
        }


        \titleditem{Update \ac{bflu}}{
                \label{fl_ela_step_3}
                Solve the \acp{bvp} given by  \crefs{eq_fl_rigid_fl_1_reference_disc_aux,eq_poiss_like_phi} and \acp{bc}
                \crefs{bc_fl_rigid_1_reference_disc_aux,bc_fl_rigid_2_reference_disc_aux,bc_fl_rigid_3_reference_disc_aux,bc_fl_rigid_5_reference_disc_aux,bc_phi_1,bc_phi_2} and
                \be
                \label{bc_fl_ela_4_current_disc}
                \bvfr = \bueladot  \text{ on }  \pomcircouteqr,
                \ee
                which replaces \cref{bc_fl_rigid_5_reference_disc_aux} for a rigid body, and solve \cref{eq_phi_1}.


        }\end{itemize}



An example of the coupled dynamics of \ac{bflu} and \ac{ela} is shown in \cref{fig_fluid_ela}.
Here, $\pomcircouteqr$ is an ellipse with semi-axes $a = 0.2\, \met$, $b = 0.1\, \met$ and center located at  $\bxr = (0.4 \, \met, 0.2\, \met)$, and $\pomcircineqr$ has radius $r = 0.0125 \, \met$ and center located at $\bxr$.
The \ac{ela} density and elastic moduli are, respectively, $\rhoela = 10^3 \,\kg/\met^2$ and $\kel = \muel = 10\,  \kg/\sec^2$. The other parameters are the same as in \cref{fig_fluid_rigid}.
% These parameters  correspond approximately to ...


\begin{figure*}
        \centering
        \includegraphics[width=\textwidth]{figures/figure_3/figure_3.pdf}
        \caption{
                \label{fig_fluid_ela}
                Interaction between a \acl{bflu} and an \acl{ela}. The notation is the same as in \cref{fig_fluid_rigid}, and the \acl{ela} is depicted as a red mesh.
        }
\end{figure*}
